%%
%% This is file `sample-sigconf.tex',
%% generated with the docstrip utility.
%%
%% The original source files were:
%%
%% samples.dtx  (with options: `sigconf')
%% 
%% IMPORTANT NOTICE:
%% 
%% For the copyright see the source file.
%% 
%% Any modified versions of this file must be renamed
%% with new filenames distinct from sample-sigconf.tex.
%% 
%% For distribution of the original source see the terms
%% for copying and modification in the file samples.dtx.
%% 
%% This generated file may be distributed as long as the
%% original source files, as listed above, are part of the
%% same distribution. (The sources need not necessarily be
%% in the same archive or directory.)
%%
%% The first command in your LaTeX source must be the \documentclass command.
\documentclass[acmsmall]{acmart}

% TODO remove XD
\usepackage{xcolor}
\newcommand{\secfunc}[1]{{\color{magenta}#1}}
\newcommand{\mention}[1]{{\color{cyan}#1}}
\newcommand{\plan}[1]{{\color{purple}#1}}
\newcommand{\bp}[1]{{\color{violet}#1}}
\newcommand{\draft}[1]{{\color{blue}#1}}
\newcommand{\review}[1]{{\color{black}#1}}
\newcommand{\todo}[1]{{\color{orange}#1}}
%%
%% \BibTeX command to typeset BibTeX logo in the docs
\AtBeginDocument{%
  \providecommand\BibTeX{{%
    \normalfont B\kern-0.5em{\scshape i\kern-0.25em b}\kern-0.8em\TeX}}}

%% Rights management information.  This information is sent to you
%% when you complete the rights form.  These commands have SAMPLE
%% values in them; it is your responsibility as an author to replace
%% the commands and values with those provided to you when you
%% complete the rights form.
\setcopyright{acmcopyright}
\copyrightyear{2018}
\acmYear{2018}
\acmDOI{10.1145/1122445.1122456}

%% These commands are for a PROCEEDINGS abstract or paper.
\acmConference[Woodstock '18]{Woodstock '18: ACM Symposium on Neural
  Gaze Detection}{June 03--05, 2018}{Woodstock, NY}
\acmBooktitle{Woodstock '18: ACM Symposium on Neural Gaze Detection,
  June 03--05, 2018, Woodstock, NY}
\acmPrice{15.00}
\acmISBN{978-1-4503-9999-9/18/06}


%%
%% Submission ID.
%% Use this when submitting an article to a sponsored event. You'll
%% receive a unique submission ID from the organizers
%% of the event, and this ID should be used as the parameter to this command.
%%\acmSubmissionID{123-A56-BU3}

%%
%% The majority of ACM publications use numbered citations and
%% references.  The command \citestyle{authoryear} switches to the
%% "author year" style.
%%
%% If you are preparing content for an event
%% sponsored by ACM SIGGRAPH, you must use the "author year" style of
%% citations and references.
%% Uncommenting
%% the next command will enable that style.
%%\citestyle{acmauthoryear}

%%
%% end of the preamble, start of the body of the document source.
\begin{document}

%%
%% The "title" command has an optional parameter,
%% allowing the author to define a "short title" to be used in page headers.
\title{Analyzing Build Logs Using Programming by Example}

\author{Carolin Brandt}

%%
%% By default, the full list of authors will be used in the page
%% headers. Often, this list is too long, and will overlap
%% other information printed in the page headers. This command allows
%% the author to define a more concise list
%% of authors' names for this purpose.
\renewcommand{\shortauthors}{Brandt, et al.}

%%
%% The abstract is a short summary of the work to be presented in the
%% article.
\begin{abstract}
  ...
\end{abstract}

%%
%% The code below is generated by the tool at http://dl.acm.org/ccs.cfm.
%% Please copy and paste the code instead of the example below.
%%
%\begin{CCSXML}
%<ccs2012>
% <concept>
%  <concept_id>10010520.10010553.10010562</concept_id>
%  <concept_desc>Computer systems organization~Embedded systems</concept_desc>
%  <concept_significance>500</concept_significance>
% </concept>
% <concept>
%  <concept_id>10010520.10010575.10010755</concept_id>
%  <concept_desc>Computer systems organization~Redundancy</concept_desc>
%  <concept_significance>300</concept_significance>
% </concept>
% <concept>
%  <concept_id>10010520.10010553.10010554</concept_id>
%  <concept_desc>Computer systems organization~Robotics</concept_desc>
%  <concept_significance>100</concept_significance>
% </concept>
% <concept>
%  <concept_id>10003033.10003083.10003095</concept_id>
%  <concept_desc>Networks~Network reliability</concept_desc>
%  <concept_significance>100</concept_significance>
% </concept>
%</ccs2012>
%\end{CCSXML}
%
%\ccsdesc[500]{Computer systems organization~Embedded systems}
%\ccsdesc[300]{Computer systems organization~Redundancy}
%\ccsdesc{Computer systems organization~Robotics}
%\ccsdesc[100]{Networks~Network reliability}

%%
%% Keywords. The author(s) should pick words that accurately describe
%% the work being presented. Separate the keywords with commas.
\keywords{ci, buildlogs, programming by example}

%%
%% This command processes the author and affiliation and title
%% information and builds the first part of the formatted document.
\maketitle


\secfunc{section function - \textbackslash secfunc} \\
\mention{things to mention / reference - \textbackslash mention} \\
\plan{what to write here - \textbackslash plan} \\
\bp{actual bullet points - \textbackslash bp} \\
\draft{final text drafty - \textbackslash draft} \\
\review{final text review ready - \textbackslash review} \\
\todo{ToDo - \textbackslash todo} \\

\section{Introduction}
\subsection{Motivation}
Many software projects now use continious integration (CI) to improve \mention{reasons to do this, look into Proksch papers}. Theses CI builds often produce very long and verbose build logs \mention{log characteristics, cite what Moritz cited in his proposal?}, stating the progress and results of the various steps within the build.

These build logs are a highly valuable data source. First of all, for the developers that read them to analyze why their build failed or .. \mention{find more reasons}. Second, for reasearchers that can harvest the information contained in the logs (and their metadata) to study the software engineering process of a project. However they can only harvest the information within the build logs if they can adequately extract the information relevant to them.

There are many different extraction techniques use for this task. Beller et al.  used regular expressions when analysing the failure reasons of ruby and Java Maven buildlogs from TravisCI \cite{beller2017oops}, while Vassallo et al. wrote a custom parser for Java Maven buildlogs to gather information for build repair hints \cite{vassallo2018un-break}. Recently Amar et al. greatly reduced the number of build log lines for a developer to inspect by creating a diff between the logs from a failed and successful build.~\cite{amar2019mining} \bp{anecdotal: keyword search?}. Apart from those there are various more extraction techniques like searching for keywords, ... \todo{more?} \bp{problem which to choose}

With our work we want to support developers, researchers or project managers in deciding which technique is the best one for their use case.
\todo{describe what we do and our results!}
\subsection{Contribution}
\begin{itemize}
  \item[RQ1:] What criteria influence the suitability of an information extraction technique for CI build logs?
  \item[RQ2:] Is Programming by Example suited to extract information from CI build logs?
  \item[RQ3:] When are text similarity or keyword search better suited for information extraction from CI build logs?
\end{itemize}
\subsection{Overview}
This paper first presents an overview over research works related to our topic. Then we characterize build logs and different techniques used for extracting information from them. We explain our three focus techniques for this work, namely PROSE regular expression program synthesis by example, text similarity and simple keyword search.
Further, Section \ref{data-set} presents our data set and the labeling and validation process. Section \ref{study} describes our study comparing our three focus techniques.

\section{Related Work}
similar to before:
\subsection{Build Log Analysis Tools}
\subsection{Continuous Integration}
\subsection{Information Extraction and Retrieval}
\subsection{Runtime Log Parsing}
\subsection{Semi-Structured Data?}

%\section{Method}
\section{Characterizing Build Log Information Extraction Techniques}
\subsection{Characteristics of a Build Log}
describe semi-strutcturedness, separate from runtime logs?
\subsubsection{Extractable Data in Build Logs}
leave out?
\subsection{Model for Information Extraction Techniques}
picture from thesis: configuration, output accuracy, ...
\subsection{Program Synthesis by Example}
\subsection{Text Similarity}
\subsection{Keyword Search}

\section{The Failing Build Logs Data Set}
\label{data-set}
\subsection{Data Structure / Points}
\subsection{Build Log Collection}
\subsection{Validation of Labeled Data}

\section{Technique Comparison Study}
\label{study}
evaluations run \& graphs \& data
\subsection{Study Design}

\subsection{Study Execution}

\subsection{Results}

\subsection{Discussion}
comparing extraction techniques,e
\subsection{Threats to Validity}

\section{Conclusion}

\subsection{Future Work}

%%
%% The acknowledgments section is defined using the "acks" environment
%% (and NOT an unnumbered section). This ensures the proper
%% identification of the section in the article metadata, and the
%% consistent spelling of the heading.
\begin{acks}
To Robert, for the bagels and explaining CMYK and color spaces.
\end{acks}

%%
%% The next two lines define the bibliography style to be used, and
%% the bibliography file.
\bibliographystyle{ACM-Reference-Format}
\bibliography{paper}

%%
%% If your work has an appendix, this is the place to put it.
\appendix

\section{Research Methods}

\subsection{Part One}


\end{document}
\endinput
%%
%% End of file `sample-sigconf.tex'.
