%%
%% This is file `sample-sigconf.tex',
%% generated with the docstrip utility.
%%
%% The original source files were:
%%
%% samples.dtx  (with options: `sigconf')
%% 
%% IMPORTANT NOTICE:
%% 
%% For the copyright see the source file.
%% 
%% Any modified versions of this file must be renamed
%% with new filenames distinct from sample-sigconf.tex.
%% 
%% For distribution of the original source see the terms
%% for copying and modification in the file samples.dtx.
%% 
%% This generated file may be distributed as long as the
%% original source files, as listed above, are part of the
%% same distribution. (The sources need not necessarily be
%% in the same archive or directory.)
%%
%% The first command in your LaTeX source must be the \documentclass command.
\documentclass[sigconf]{acmart}

% TODO remove XD
\usepackage{xcolor}
\newcommand{\secfunc}[1]{{\color{magenta}#1}}
\newcommand{\mention}[1]{{\color{cyan}#1}}
\newcommand{\plan}[1]{{\color{purple}#1}}
\newcommand{\bp}[1]{{\color{violet}#1}}
\newcommand{\draft}[1]{{\color{blue}#1}}
\newcommand{\review}[1]{{\color{black}#1}}
\newcommand{\todo}[1]{{\color{orange}#1}}
%%
%% \BibTeX command to typeset BibTeX logo in the docs
\AtBeginDocument{%
  \providecommand\BibTeX{{%
    \normalfont B\kern-0.5em{\scshape i\kern-0.25em b}\kern-0.8em\TeX}}}

%% Rights management information.  This information is sent to you
%% when you complete the rights form.  These commands have SAMPLE
%% values in them; it is your responsibility as an author to replace
%% the commands and values with those provided to you when you
%% complete the rights form.
\setcopyright{acmcopyright}
\copyrightyear{2018}
\acmYear{2018}
\acmDOI{10.1145/1122445.1122456}

%% These commands are for a PROCEEDINGS abstract or paper.
\acmConference[Woodstock '18]{Woodstock '18: ACM Symposium on Neural
  Gaze Detection}{June 03--05, 2018}{Woodstock, NY}
\acmBooktitle{Woodstock '18: ACM Symposium on Neural Gaze Detection,
  June 03--05, 2018, Woodstock, NY}
\acmPrice{15.00}
\acmISBN{978-1-4503-9999-9/18/06}


%%
%% Submission ID.
%% Use this when submitting an article to a sponsored event. You'll
%% receive a unique submission ID from the organizers
%% of the event, and this ID should be used as the parameter to this command.
%%\acmSubmissionID{123-A56-BU3}

%%
%% The majority of ACM publications use numbered citations and
%% references.  The command \citestyle{authoryear} switches to the
%% "author year" style.
%%
%% If you are preparing content for an event
%% sponsored by ACM SIGGRAPH, you must use the "author year" style of
%% citations and references.
%% Uncommenting
%% the next command will enable that style.
%%\citestyle{acmauthoryear}

%%
%% end of the preamble, start of the body of the document source.
\begin{document}

%%
%% The "title" command has an optional parameter,
%% allowing the author to define a "short title" to be used in page headers.
\title{Analyzing Build Logs Using Programming by Example}

\author{Carolin Brandt}

%%
%% By default, the full list of authors will be used in the page
%% headers. Often, this list is too long, and will overlap
%% other information printed in the page headers. This command allows
%% the author to define a more concise list
%% of authors' names for this purpose.
\renewcommand{\shortauthors}{Brandt, et al.}

%%
%% The abstract is a short summary of the work to be presented in the
%% article.
\begin{abstract}
  ...
\end{abstract}

%%
%% The code below is generated by the tool at http://dl.acm.org/ccs.cfm.
%% Please copy and paste the code instead of the example below.
%%
%\begin{CCSXML}
%<ccs2012>
% <concept>
%  <concept_id>10010520.10010553.10010562</concept_id>
%  <concept_desc>Computer systems organization~Embedded systems</concept_desc>
%  <concept_significance>500</concept_significance>
% </concept>
% <concept>
%  <concept_id>10010520.10010575.10010755</concept_id>
%  <concept_desc>Computer systems organization~Redundancy</concept_desc>
%  <concept_significance>300</concept_significance>
% </concept>
% <concept>
%  <concept_id>10010520.10010553.10010554</concept_id>
%  <concept_desc>Computer systems organization~Robotics</concept_desc>
%  <concept_significance>100</concept_significance>
% </concept>
% <concept>
%  <concept_id>10003033.10003083.10003095</concept_id>
%  <concept_desc>Networks~Network reliability</concept_desc>
%  <concept_significance>100</concept_significance>
% </concept>
%</ccs2012>
%\end{CCSXML}
%
%\ccsdesc[500]{Computer systems organization~Embedded systems}
%\ccsdesc[300]{Computer systems organization~Redundancy}
%\ccsdesc{Computer systems organization~Robotics}
%\ccsdesc[100]{Networks~Network reliability}

%%
%% Keywords. The author(s) should pick words that accurately describe
%% the work being presented. Separate the keywords with commas.
\keywords{ci, buildlogs, programming by example}

%%
%% This command processes the author and affiliation and title
%% information and builds the first part of the formatted document.
\maketitle


\secfunc{section function - secfunc} \\
\mention{things to mention / reference - mention} \\
\plan{what to write here - plan} \\
\bp{actual bullet points - bp} \\
\draft{final text drafty - draft} \\
\review{final text review ready - review} \\
\todo{ToDo - todo} \\

\section{Introduction}
\secfunc{why? what? research questions? purpose? hypothesis? wide $\rightarrow$ narrow}

\draft{Fixing a broken continuous integration build has a high priority for developers~\cite{vassallo2018un-break}, therefore decreasing the time until the green build state is restored improves the overall productivity of developers.
During this assignment an engineer spends most of their time reading very long and verbose build logs, trying to recover the relevant information from between many uninteresting other lines.

One way to support them would be to extract the desired Information and present it in and present it in a structured way.
Usually such functionality is realized by tools that employ handrcrafted regular expressions to find the desired text parts within the build log.
However, creating these regular expressions is a tedious and mentally complicated work, especially when they have to be maintained some months in the future \textemdash \ reunderstanding old regular expressions is known to be a dreading task for developers \textemdash.
Our goal is it to take this burden of programmers by enabling them to define extraction programs through simply giving examples of the desired extractions.

We implemented \todo{<toolname, ALBE?, Analyzing Logs By Example>}, a prototype for defining text extractions from build logs through examples using the Microsoft PROSE library~\cite{le2014flashextract:}.
After collecting a diverse set of build logs from Travis CI we designed a meta-model of the contained information.
Our evaluation of ALBE showed that it is able to define extractinos for various desired informations, that it shortens development time in comparison to manual regex cnosturction and ...
}

\section{Related Work}

\subsection{Continuous Integration}

\draft{
The software engineering research community started to look into build logs as continuous integration becomes a vital part of a modern software engineering process. Reserachers have analyzed industrial and open source logs for failure reasons and their impact on development.

Seo et al.~\cite{seo2014programmers} found that at google few error types such as dependency mismatches are the most prominent cause of build failures and most failures are resolved within two builds. Vassallo et al.~\cite{vassallo2017a-tale} compared open source projects in java to industrial ones. They determined that testing failures outweigh compilation errors and that open source bulids fail most often because of unit tests, whereas release preparations is the primary cause in industrial projects. Beller et al.~\cite{beller2017oops} showed that testing is central to continuous integration when evaluating Travis CI logs for Java and ruby builds. They observed very different kinds, failure rates and numbers of test between programming languages and explained that the low failure rates hint at code being pretested before sent to the CI server.

All these researchers described building parsers in order to evaluate the studied build logs. Our work could support their research by easing the parser development and enable them to cover more languages and build tools easier.
}

\subsection{Tool}

\draft{
Vassallo et al.~\cite{vassallo2018un-break} also made it their task to shortening the time it takes developers to understand builogs. They summarize relevant information in Java/maven buildlogs and augment them with links to related stack overflow posts. They observed that highlighting the locality and context of an issue is helpful to programmers. We strive to enable a similar summarization by text extraction while also covering a wider array of programming languages.

Amar et al.~\mention{cite icse paper} reduced the lines of a log to be inspected by the engineer through removing lines that appear both in passing and failing build logs. Further they employ information retrieval techniques to identify the lines most likely hinting at the cause of the error. In contrast to that, instead of comparing whole our tool ALBE extracts specific parts of the build logs. As this is mostly dependent on the implicit reoccurring structure within the logs we operate on the full log output.
}


\subsection{Information Extraction and Retrieval}

\plan{Le et al. developed FlashExtract as part of the PROSE Framework, tool to synthesize programs based on a few given examples, extract substrings from semi-structured texts, users can extract multiple fields and structure them with hierarchy and sequence, uses \todo{inductive - explain or throw out} synthesis algorithm to synthesize intended program, eliminates the need for the user to understand structure of entire document or to be able to code
	
we are applying their library to the domain of buildlogs, using programming by example to take away the need to tediously develop and maintain the regular expressions, traditionally used in information extraction
}

\mention{PROSE fashextract}

\mention{queriying semi structured data}

\plan{characteristics of semi-structured data (structure irregular, implicit, partial, rapidly evolving) and explaining why this applies to buildlogs, extract information to build a structured layer above the unformed data for efficient data access (for developers here! -> impact on our work with buildlogs) \cite{abiteboul1997querying}}

\mention{sth about IR / paragraph retrieval}
\plan{practices from information retrieval, cutting into lines or paragraphs, calculating similarities through methods xy}

\plan{difference of us to IR (imprecise \& rough)}

\section{Method}
\secfunc{how? when? material? narrow}

\todo{want?: flow of research figure}

\todo{study design vs. study carry out}

\subsection{Data collection}
\plan{describe process of collecting buildlog files, sampling}

\bp{querying ghtorrent and travis api with (insert fancy name for data collection tool) for x buildlogs of each status of the top-watched repositories on github that also use travis ci}

\subsection{Meta-Model}
\bp{through manual examination of y of the buildlogs we collected designed a metamodell  of the contained/extractable/useful/interesting information}

\plan{figure with metamodel}

\plan{explain (shortly? partly?) meta model classes}

\subsection{Our Tool}
\plan{interaction and usage and output of tool, maybe integration possibilities}

\plan{capabilities: give example extractions \& limitations}

\plan{necessary or interesting implementation details}

\section{Results}
\secfunc{hard numbers! answers to research questions, narrow}

\plan{achieved accuracy, samples needed for same accuracy as related tools (travistorrent), aging of learned programs with newer data}

\section{Analysis / Discussion}
\secfunc{interpretation, implication of answers and why does it matter, comparison to previous findings, narrow $\rightarrow$ wide}

\plan{our tool works well for defined cases, limitations of tool/prose/regex usage in general, every x weeks extraction has to be rewritten to keep accuracy, well improved if only y new examples have to be added}

\section{Conclusion and Future Work}
\secfunc{summarize, results again, future research}

\plan{repeat everything and results}

\plan{future}

\bp{make avaliable to other researchers

integrate into travis

\plan{as always} extend evaluation and data collection

use dataset to evaluate against existing approaches

}

%%
%% The acknowledgments section is defined using the "acks" environment
%% (and NOT an unnumbered section). This ensures the proper
%% identification of the section in the article metadata, and the
%% consistent spelling of the heading.
\begin{acks}
To Robert, for the bagels and explaining CMYK and color spaces.
\end{acks}

%%
%% The next two lines define the bibliography style to be used, and
%% the bibliography file.
\bibliographystyle{ACM-Reference-Format}
\bibliography{paper}

%%
%% If your work has an appendix, this is the place to put it.
\appendix

\section{Research Methods}

\subsection{Part One}


\end{document}
\endinput
%%
%% End of file `sample-sigconf.tex'.
