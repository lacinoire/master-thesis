%%
%% This is file `sample-sigconf.tex',
%% generated with the docstrip utility.
%%
%% The original source files were:
%%
%% samples.dtx  (with options: `sigconf')
%% 
%% IMPORTANT NOTICE:
%% 
%% For the copyright see the source file.
%% 
%% Any modified versions of this file must be renamed
%% with new filenames distinct from sample-sigconf.tex.
%% 
%% For distribution of the original source see the terms
%% for copying and modification in the file samples.dtx.
%% 
%% This generated file may be distributed as long as the
%% original source files, as listed above, are part of the
%% same distribution. (The sources need not necessarily be
%% in the same archive or directory.)
%%
%% The first command in your LaTeX source must be the \documentclass command.
\documentclass[sigconf]{acmart}

% TODO remove XD
\definecolor{ballblue}{rgb}{0.13, 0.67, 0.8}
\usepackage[color=ballblue]{todonotes}

%%
%% \BibTeX command to typeset BibTeX logo in the docs
\AtBeginDocument{%
  \providecommand\BibTeX{{%
    \normalfont B\kern-0.5em{\scshape i\kern-0.25em b}\kern-0.8em\TeX}}}

%% Rights management information.  This information is sent to you
%% when you complete the rights form.  These commands have SAMPLE
%% values in them; it is your responsibility as an author to replace
%% the commands and values with those provided to you when you
%% complete the rights form.
\setcopyright{acmcopyright}
\copyrightyear{2018}
\acmYear{2018}
\acmDOI{10.1145/1122445.1122456}

%% These commands are for a PROCEEDINGS abstract or paper.
\acmConference[Woodstock '18]{Woodstock '18: ACM Symposium on Neural
  Gaze Detection}{June 03--05, 2018}{Woodstock, NY}
\acmBooktitle{Woodstock '18: ACM Symposium on Neural Gaze Detection,
  June 03--05, 2018, Woodstock, NY}
\acmPrice{15.00}
\acmISBN{978-1-4503-9999-9/18/06}


%%
%% Submission ID.
%% Use this when submitting an article to a sponsored event. You'll
%% receive a unique submission ID from the organizers
%% of the event, and this ID should be used as the parameter to this command.
%%\acmSubmissionID{123-A56-BU3}

%%
%% The majority of ACM publications use numbered citations and
%% references.  The command \citestyle{authoryear} switches to the
%% "author year" style.
%%
%% If you are preparing content for an event
%% sponsored by ACM SIGGRAPH, you must use the "author year" style of
%% citations and references.
%% Uncommenting
%% the next command will enable that style.
%%\citestyle{acmauthoryear}

%%
%% end of the preamble, start of the body of the document source.
\begin{document}

%%
%% The "title" command has an optional parameter,
%% allowing the author to define a "short title" to be used in page headers.
\title{Analyzing Buildlogs Using Programming by Example}

\author{Carolin Brandt}

%%
%% By default, the full list of authors will be used in the page
%% headers. Often, this list is too long, and will overlap
%% other information printed in the page headers. This command allows
%% the author to define a more concise list
%% of authors' names for this purpose.
\renewcommand{\shortauthors}{Brandt, et al.}

%%
%% The abstract is a short summary of the work to be presented in the
%% article.
\begin{abstract}
  ...
\end{abstract}

%%
%% The code below is generated by the tool at http://dl.acm.org/ccs.cfm.
%% Please copy and paste the code instead of the example below.
%%
%\begin{CCSXML}
%<ccs2012>
% <concept>
%  <concept_id>10010520.10010553.10010562</concept_id>
%  <concept_desc>Computer systems organization~Embedded systems</concept_desc>
%  <concept_significance>500</concept_significance>
% </concept>
% <concept>
%  <concept_id>10010520.10010575.10010755</concept_id>
%  <concept_desc>Computer systems organization~Redundancy</concept_desc>
%  <concept_significance>300</concept_significance>
% </concept>
% <concept>
%  <concept_id>10010520.10010553.10010554</concept_id>
%  <concept_desc>Computer systems organization~Robotics</concept_desc>
%  <concept_significance>100</concept_significance>
% </concept>
% <concept>
%  <concept_id>10003033.10003083.10003095</concept_id>
%  <concept_desc>Networks~Network reliability</concept_desc>
%  <concept_significance>100</concept_significance>
% </concept>
%</ccs2012>
%\end{CCSXML}
%
%\ccsdesc[500]{Computer systems organization~Embedded systems}
%\ccsdesc[300]{Computer systems organization~Redundancy}
%\ccsdesc{Computer systems organization~Robotics}
%\ccsdesc[100]{Networks~Network reliability}

%%
%% Keywords. The author(s) should pick words that accurately describe
%% the work being presented. Separate the keywords with commas.
\keywords{ci, buildlogs, programming by example}

%%
%% This command processes the author and affiliation and title
%% information and builds the first part of the formatted document.
\maketitle


\section{Introduction}

ci important for sw projects,

broken build stops dev process until its fixed, 

shortening fixing time important to increase dev productivity,
 
to detect underlying failure huge buildlog has to be read, finding relevant information between a lot of lines unimportant for that task, 

extracting the dev-relevant information from logs and present only them in a structured way could be key improvement on dev productivity, 

traditional: regexes built for exact information extraction

hard to maintain, regex understanding effort, don't generalize / have to be redone when a bulidlog changes / language or build tool is switched, 

... our approach ..., 

... paper structure ...

\section{Related Work}
\paragraph{CI Research}
- se research community is starting to look into builodlogs

- paper at google: ...et al., large study at google, quantitative analysis of which errors appear during builds at google, java and c++ projects, build parser for error messages in logs, results: very few types of errors make up most of the failures, most dependency issues or type missmatchs, (experienced developers have similar failure rates to all other devs?), most errors fixed in 2 build iterations, emphasize that more specific failure messages \& fix proposals would be valuable,
... compared to us: we could support this research by enabling easier data extraction form the analyzed logs, tool could also be used by google for more concise information extracted form the logs and given to the user directly → faster failure resolvement \cite{seo2014programmers},

- ing vs. oss (java): Vasssalo et al., comparing builds from java oss projects and ING Nederland, borad taxonomy and anlaysis of failures during the ci process, oss mainly falis due to unit testing, Ing / industrial ... / bank / mainly duriing release preparation phase  → oss and industrial veeery different

... compared to our approach: flexible enough for different kind of logs e.g. oss and ing → again, helping research \cite{vassallo2017a-tale}, 


- Beller et al., largescale analyzis of over 2 Mio buildlogs of oss in java and ruby obtained from travis ci, tests are central part of ci and main reason for builds to fail, ci adoption \& influence of failed tests orthogonal to chosen programming language, build time main factor in feedback delay (useful to mention in our paper? we minimize more at another stage: understanding)

... compared to our approach: development custom log analysis for build failures specific for java and ruby. With pbe we could replace / simplyfy the manual regex construction\ \cite{beller2017oops}

\todo[inline]{might be far too long... compress \& talk about differentiation to our work once}



\paragraph{Analyzing Buildlogs}
Vassallo et al., goal: support developer in finding reasons for a build break fast, developed tool: analyze buildlogs, summarize, link realated stack overflow discussions (thesis: created metamodel of bulildlog structure for maven, contains references to original output of build stages, built specific parser for these logs + the metamodel, tool BART (build abstraction and recovery tool) works with hint generators that operate on the metamodel instantiation to not have every hint generator parse the whole log again), in their evaluation found that "dependency breaks and testing failures seem to be easy to understand" and that a good summary highlighting the locality of the issues seems to be an important factor of fixing time. \cite{vassallo2018un-break}

... compared to our work: they focused on java/maven, we want to explore supporting a wide variety of buildlogs. our metamodel inspired by them, our pbe tool should replace their not closer described "parser" to fill a metamodel instantiation

\paragraph{Information Extraction from semi-structured data using program synthesis}
- Le et al. developed FlashExtract, a framework to extract revelanvt data from semi-structured documents using examples, users can extract multiple fields and structure them with hierachy and sequence, uses inductive synthesis algorithm to synthesize intended program, domain-independent approach, eliminates the need for the user to understand structure of entire document or to be able to code (yes, developers can code, but still effort \& maintenance of old code difficult)

... compared to our approach: we use exactly their framework for our information extraction, apply it to the domain of buildlogs, MAYBE: adopt internals (ranking, if explained before here) to the specifics of the use case buildlog 

\todo[inlince]{stuff below maybe in theoretical background better}
technical: extract by generating various programs for slicing substrings, either absolute positions or regexes for slicing boundaries, regexes with lookahead/lookbehind, various programs which comply with examples are generated, ranked by how likely they are intended, make example: regex more likely than absolute positions for slicing boundaries

- include \cite{smith1997information}? other specific approaches of information extraction from semi-structured data? alternatively: rename to program synthesis

for that paper: extract information from semi-structured text documents, exemplary french govermental documents \todo{check that that is correct}, divide into "contexts" by headers / subsections, extract desired "concepts" from specified contexts via possible start regex and keywords, end by next concept starting or end regex, \todo{do they have an evaluation?}

\paragraph{Information retrieval}
realted topic: Getting Information out of (semi-structured) data. A lot of focus on NLP and text summarization or similarity measurement.

Differentiation: we do not compare different documents -> no use for similarity computations, programmers expect precise information: (TODO why?) rough summarization of buildlogs often uninteresting: a lot of repeated parts that are in all bulidlogs. For developer mostly one specific information (why did it fail?) and maybe supporting data (stacktraces, error messages) relevant. If we want to incorporate our results into another tool (travis button with failed test, predictive CI) they need to be \textbf{exact}. We are not trying to generally summarize bulidlogs here, we are trying to replace regular expression manual creation by pbe.

\section{Background}

\paragraph{semi-structured data}
characteristics of semi-structured data (structure irregular, implicit, partial, rapidly evolving) and explaining why this applies to buildlogs, extract information to build a structured layer above the unformed data for efficient data access (for developers here! -> impact on our work with buildlogs) \cite{abiteboul1997querying}

maybe: reference \cite{smith1997information} and explain approach of first structuring into sections / "contexts" before extracting acutal information. but only if we actually do that and not just always parse the whole log.

program synthesis from example, neccessary theoretical stuff,
here all the references to prose theoretical papers \cite{le2014flashextract:}, \cite{polozov2015flashmeta:}, \cite{mitchell1982generalization}, 

\section{Data collection}
describe process of collecting buildlog files, sampling

\section{Meta-Model}
describe metamodel of data obtainable from buildlogs

\section{Our Tool}

\section{Evaluation}

\section{Conclusion and Future Work}

%%
%% The acknowledgments section is defined using the "acks" environment
%% (and NOT an unnumbered section). This ensures the proper
%% identification of the section in the article metadata, and the
%% consistent spelling of the heading.
\begin{acks}
To Robert, for the bagels and explaining CMYK and color spaces.
\end{acks}

%%
%% The next two lines define the bibliography style to be used, and
%% the bibliography file.
\bibliographystyle{ACM-Reference-Format}
\bibliography{paper}

%%
%% If your work has an appendix, this is the place to put it.
\appendix

\section{Research Methods}

\subsection{Part One}


\end{document}
\endinput
%%
%% End of file `sample-sigconf.tex'.
