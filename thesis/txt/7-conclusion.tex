

\providecommand{\myrootdir}{..}
\documentclass[\myrootdir/main.tex]{subfiles}

\begin{document}

\chapter{Conclusion and Future Work}
\label{sec:conclusion-fw}
The goal of this thesis is to support researches and developers when they decide how to analyze build logs.
We implemented and compared three different chunk retrieval techniques on our \emph{LogChunks} data set.
Our results show that the structural representation of the targeted information in the build logs is the main factor to consider when choosing a suitable technique.
Secondary factors are confidence in the produced output and whether precision or recall are more important for the task at hand.

In the following we describe various future research opportunities based on our work:
\begin{itemize}
  \item \textbf{Further Analysis of \emph{LogChunks}} We created the \emph{LogChunks} data set specifically for our comparison study, though it can be the basis for various further analyses of build log data.
  The keywords can be investigated to answer which keywords do developers use to search for the reason the build failed within build logs.
  \item \textbf{Cross-Repository Build Log Analysis} We trained and tested each chunk retrieval technique on examples from one repository.
  We propose to analyze how techniques could be trained across repositories, building the cornerstones for build environment agnostic analysis tools.
  \item \textbf{Comparison with more Chunk Retrieval Techniques} This thesis investigates the three chunk retrieval techniques PBE, CTS and KWS. more techniques (which kinds of techniques? express in model! e.g.\ diffs might not work?), same evaluation, ir automated keyword search/expansion
  \item \textbf{Refinement of Retrieval Quality for each Technique} We investigated basic configurations existing techniques applied to chunk retrieval from build logs.
  In a next step, each of these techniques could be refined and configured to better approach the domain of build logs.
  The \emph{LogChunks} data set and our study results act as a baseline to evaluate further technique improvements.
  We propose the following improvements:
    \begin{itemize}
      \item \textbf{Custom Ranking and Tokens for PBE} The program synthesis through PROSE ranks possible programs according to what the user most likely intended.
      The ranking rules provided by the FlashExtract DSL could be adapted to fit common build log chunk retrieval tasks.
      FlashExtract includes special tokens when enumerating possible regular expressions.
      These could be extended with tokens found in build logs, such as ``-'',``='',``ERROR'' or ``[OK''.
      \item \textbf{Meta-Parameter Optimization for CTS} Information retrieval techniques have various meta-parameters which can be optimized for the specific use case~\cite{panichella2016parameterizing}.
      We propose to further investigate improvements in preprocessing of the log text, tokenization of the log lines into terms and stop words lists.
    \end{itemize}
  \item \textbf{Usability Analysis of Chunk Retrieval Output} Our analysis of the output produced by the chunk retrieval focussed on precision and recall.
  In addition to that, we propose to investigate how useful these outputs are to developers through controlled experiments.
\end{itemize}

\end{document}
