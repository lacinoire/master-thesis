
\providecommand{\myrootdir}{..}
\documentclass[\myrootdir/main.tex]{subfiles}

\begin{document}

\chapter{Chunk Retrieval Techniques for Build Logs}
\label{sec:techniques}
This chapter introduces the concept of chunk retrieval techniques.
We use these techniques to extract text chunks from build logs that represent a specific, targeted information.
This chapter first presents how a build log, which we take as input to the chunk retrieval, is created by a continuous integration (CI) build.
Next, the chapter describes retrievable information chunks in build logs and gives examples found in Travis CI build logs.
We illustrate why it is useful to extract the presented information chunks.
The following section introduces our concept of chunk retrieval techniques.
We present the three chunk retrieval techniques we investigate in this thesis:
program synthesis by example (PBE), common text similarity (CTS) and keyword search (KWS).

\begin{figure}[htbp]
	\centering
	\includegraphics[width=\textwidth, clip]{img/build-overview.pdf}
	\caption{The different entities related to a CI build.}
	\label{fig:build-overview}
\end{figure}

\section{Characteristics of a Build Log}
\label{sec:bl-characteristics}
The idea of CI is to catch errors early by integrating software changes fast and often~\cite{humble2010continuous}.
Companies link a CI server, e.g.\ Travis CI, to their source code repository.
After making a change, the developer commits and pushes the new version of the code to the repository.
A push on specific branches or the creation of a pull request triggers a CI build.

A CI build typically runs through the following stages:
\begin{itemize}
	\item Pulling the new, changed version of the source code into the build environment.
	\item Building the software, i.e.\ compiling and packaging it~\cite{phillips2014understanding}.
	\item Running static analysis tools~\cite{zampetti2017open}.
	\item Running automated tests~\cite{beller2017oops}.
	\item Deployment of the build artifact~\cite{schermann2016empirical}.
\end{itemize}

However, these are only \emph{typical} stages and there is a high variability in the CI build processes of different software projects~\cite{staahl2014modeling}.
Some smaller projects might use CI to just ensure their code compiles as a minimal check before reviewing a pull request.
Other projects might have various stages of extensive automated testing.

Software tools involved in the build write out log messages to the console.
They communicate progress updates, error messages and warning messages to the user~\cite{yuan2012characterizing}.
We refer to to the concatenation of this output as \emph{build log}.
The structure of their output is chosen by every tool themselves.
Many have implicit or explicit structuring rules, some adhere to predefined standards like RSpec or PHPUnit~\cite{phpunit2019logging,rspec2019format}.
Figure~\ref{fig:tool-log-contribution} shows how different tools contribute to the whole build log.

\begin{figure}[htbp]
	\centering
	\includegraphics[page=1, width=0.8\textwidth, trim={5cm 0.5cm 0.5cm 0.5cm}, clip]{img/overview-graphics.pdf}
	\caption{Contribution of different tools to a build log.}
	\label{fig:tool-log-contribution}
\end{figure}

When analyzing build logs we do not necessarily have access to the exact build configuration, describing which tools are used in which order.
We also do not necessarily have access to a useable definition of the output structure of a specific tool.
Therefore, build logs are semi-structured, as described in Section~\ref{sec:rw-semi-structured-data}.


\section{Information Chunks in Build Logs}
\label{sec:bli}
CI build logs contain a great amount of information about the CI build they correspond to.
This section defines our concept of information retrievable from a CI build log.
We use the introduced terms throughout the following sections and chapters to discuss about the characteristics of chunk retrieval techniques.
To illustrate why chunk retrieval from build logs is useful, we present examples of information contained in Travis CI build logs and describe use cases for developers and researchers to retrieve them.

Central to this explanation is the piece of \emph{information} possibly retrievable from a build log.
\emph{Possibly}, because an information is not necessarily present in every build log.
If it is present, we call the text part which describes the information the \emph{information chunk} or in short \emph{chunk}.
Each chunk is always contained in a specific build log.
Figure~\ref{fig:build-overview} presents the relation between an information, a chunk and other entities involved in a CI build.

Chunks can be hierarchically ordered, determined by how their textual representations contain each other.
As this is an ad-hoc and a-posteriori structuring schema, as described in Section~\ref{sec:rw-semi-structured-data}, this section only presents hierarchical containment for chunks whose hierarchy is known.
Most chunks can appear in various hierarchical arrangements and are therefore not contained in any other chunk.

During our initial exploration and log data collection for the \emph{LogChunks} data set, we collected a broad set of build logs from 29 languages and 87 repositories from Travis CI~\cite{travisci2019webpage}.
We inspected them to get an impression of the information one would want to retrieve from a build log.
In the following, we describe examples of information chunks that can be retrieved from Travis CI build logs.
All information examples containing ``Travis'' in their name are specific to Travis CI build logs, the others can also apply to build logs from another CI environment.
Figure~\ref{fig:build-log-information} shows an overview of these information examples.

\begin{figure}[htbp]
	\centering
	\includegraphics[width=\textwidth, clip]{img/build-log-information.pdf}
	\caption{Information retrievable from build logs.}
	\label{fig:build-log-information}
\end{figure}

\begin{itemize}
	\item \textbf{BuildPhase} Build logs can be divided into the sections produced by different tools.
	      These build steps could, for example, be a \textbf{TestPhase}, \textbf{LinterPhase}, \textbf{TravisWorker} or \textbf{GitFetch}.

	\item \textbf{TravisPhase} Travis CI build logs consist of several build phases defined within the Travis CI configuration language. Within the build log each phase is framed by \lstinline{travis_fold:start:<phase name>} and \lstinline{travis_fold:end:<phase name>}.

	      A TravisPhase contains:
	      \begin{itemize}
		      \item \textbf{TravisPhaseName} The string Travis CI uses to identify the phase in start and end statements.
		      \item \textbf{TravisPhaseOutput} The output generated during the TravisPhase. The chunk presenting the \texttt{TravisPhaseOutput} is the string between the start and end statements.
	      \end{itemize}
				 Retrieving the names of the phases of a Travis CI build log could be used to later reconstruct an overview of the executed steps within a build.
	       Figure~\ref{fig:log-1} shows an example of a TravisPhase chunk and its components within a build log.

	\item \textbf{TravisTiming} Travis can measure the time of specified sections of the build process.
				A developer can retrieve this timing information to automatically monitor the performance of their build.
	      Figure~\ref{fig:log-1} shows an example of a TravisTiming chunk.
	      %It represents those by \lstinline{travis_time:start:<timing section id>} and \\ \lstinline{travis_time:end:<timing section id>:start=<start time>,finish=<finish time>,duration=<duration>}

	\item \textbf{TravisWorker} Travis CI logs which machine is executing each build.
	      Retrieving this information from multiple build logs can help visualize the impact of the build server assignment algorithm.
	      The TravisWorker is a good example that chunks belonging to the same information can have different textual representations in different build logs.
	      Figure~\ref{fig:log-0} and Figure~\ref{fig:log-2} show examples of different TravisWorker chunks.

	\item \textbf{TravisSystemInfo} At the beginning of each log Travis CI describes the tech stack of the server executing the build.
				A developer can retrieve the system information from both failing and successful logs to identify if a failure could be based on the execution environment.
	      Figure~\ref{fig:log-0} shows an example of a TravisSystemInfo chunk within a build log.

	\item \textbf{TravisTriggeringCommand} Travis CI logs the commands it uses to call certain tools.
	      These come from the \texttt{travis.yml} configuring the build.
				This information can be useful for a researcher to retrieve when they reverse engineer the build configuration.
	      Figure~\ref{fig:log-3} shows an example of a TravisTriggeringCommand chunk.

	\item \textbf{TravisExitCode} Travis CI prints all exit codes of commands.
				A researcher can retrieve these chunks to fill an overview of the build steps and why they failed.
	      Figure~\ref{fig:log-3} shows an example of a TravisExitCode chunk within a build log.

	\item \textbf{ErrorMessage} Various tools involved in the build process output messages of errors that occurred during their execution.
	      A developer can retrieve them to understand why the build failed.
	      Figure~\ref{fig:log-3} shows an example of an ErrorMessage chunk.

	\item \textbf{WarningMessage} In addition to errors, tools also print warning messages.
				Developers can collect and count them to encourage their team to resolve them.
	      Figure~\ref{fig:log-4} and Figure~\ref{fig:log-5} show two examples of WarningMessage chunks formatted differently in two build logs.

\end{itemize}

\begin{figure}[htbp]
	\centering
	\includegraphics[width=\textwidth, clip]{img/log12.png}
	\caption{Excerpt from a build log showing a TravisTiming chunk and a TravisPhase chunk, containing the TravisPhaseName chunk and the TravisPhaseOutput chunk.}
	\label{fig:log-1}
\end{figure}
\begin{figure}[htbp]
	\centering
	\includegraphics[width=\textwidth, clip]{img/log02.png}
	\caption{Excerpt from a build log showing a long TravisWorker chunk and a TravisSystemInfo chunk.}
	\label{fig:log-0}
\end{figure}
\begin{figure}[htbp]
	\centering
	\includegraphics[width=\textwidth, clip]{img/log22.png}
	\caption{Excerpt from a build log showing a short TravisWorker chunk.}
	\label{fig:log-2}
\end{figure}
\begin{figure}[htbp]
	\centering
	\includegraphics[width=\textwidth, clip]{img/log32.png}
	\caption{Excerpt from a build log showing an ErrorMessage chunk, an ExitCode chunk and a TravisTriggeringCommand chunk.}
	\label{fig:log-3}
\end{figure}


% \begin{figure}[htbp]
% 	\centering
% 	\includegraphics[width=\textwidth, clip]{img/ir-technique.pdf}
% 	\caption{Model for an Information Retrieval Technique}
% 	\label{fig:model-ie-technique}
% \end{figure}
\section{Characteristics of Chunk Retrieval Techniques}
\label{sec:blirt}
For this thesis, we want to evaluate different techniques to retrieve information chunks from build logs, which we call \emph{chunk retrieval techniques}.
The techniques we investigate do not require to parse the structure of a whole build log, but focus on extracting just one specific information.

The user provides a \textit{configuration} that specifies which information the chunk retrieval should target and supplies the necessary information for the technique to identify the targeted information chunk in a build log.
Each chunk retrieval has a specific \textit{granularity}, i.e.\ the smallest retrievable text piece and uses a specific \textit{identification technique} to select the log parts it retrieves.
Each configuration addresses a specific \textit{scope}.
The scope can be a specific project, a tool involved in the build, a programming language or global, when configuring a retrieval technique for all possible build logs.
A \emph{run} of a chunk retrieval technique consumes a build log as a plain text file and produces a string output, which consists of substrings of the build log text.

The following sections of this chapter introduce the three chunk retrieval techniques we investigate: program synthesis by example (PBE), common text similarity (CTS), and keyword search (KWS).
Lastly we describe other techniques which can also be treated as chunk retrieval techniques.
Table~\ref{tab:ctr} shows a comparison of the presented techniques.

\begin{table}[htbp]
\centering
\caption{Overview of the described chunk retrieval techniques.}
\begin{tabularx}{\textwidth}{@{}XlXlX@{}} 
\toprule
Name                         & Acronym & Identification Technique                                   & Granularity & Configuration             \\ 
\midrule
Program Synthesis by Example & PBE     & Regular expression program                                 & Character   & In/output examples      \\
Common Text Similarity       & CTS     & TF-IDF \& cosine similarity, mean number of lines expected & Line        & Output examples           \\
Keyword Search               & KWS     & Keywords, mean number of lines expected                    & Line        & Keywords, context length  \\
Random Line Retrieval        & RLR     & Random sample                                              & Line        & Retrieval length          \\
Diff Approach                & ---        & Line not present in successful log, information retrieval  & Line        & Logs from failing and successful builds      \\
\bottomrule
\end{tabularx}
\label{tab:ctr}
\end{table}

\subsection{Program Synthesis by Example (PBE)}
\label{sec:expl-pbe}
\emph{Programming by Example} is a technique which synthesizes programs according to in- and output examples provided by the user.
It enables users to create programs without a priori programming knowledge~\cite{mayer2015user}.
In the context of text extraction through regular expressions, Programming by Example relieves the developer from having to understand the whole document structure to solve a single extraction task~\cite{le2014flashextract:}.
In this work, we refer to our interpretation of Programming by Example as \emph{PBE}\@.
We investigate the suitability of PBE to retrieve information chunks from build logs.
In the following, we explain the configuration and application of PBE to chunk retrieval from CI build logs.

\paragraph{Configuration}
In/output examples are the main driver of Programming by Example.
We refer to in/output examples as \emph{examples}.
When retrieving information chunks from build logs the \emph{input} is a whole build log, i.e.\ the whole text of the build log file.
The \emph{output} is a substring of the log file text, representing the substring that should be retrieved by the synthesized program when given the corresponding input file.
One or multiple examples, the training examples, configure a chunk retrieval with PBE: they define the substring of a build log that should be extracted.
The PROSE program synthesis then tries to construct a regular expression program consistent with all training examples.
A program is consistent with an example if it returns the defined output when executed on the defined input~\cite{mitchell1982generalization}.
PBE reports an error back to the user if it could not synthesize a consistent program.
The program synthesis builds on the FlashExtract DSL, which in turn uses the FlashMeta algorithm.
Both are described in Section~\ref{sec:rw-prose}.

\paragraph{Application}
A run of PBE takes a build log file as input and applies the synthesized regular expression program.
It then returns the substring of the build log matched by the program or an empty string if the program found no match.

\subsection{Common Text Similarity (CTS)}
\label{sec:expl-ts}
Text Similarity approaches are used to filter unstructured textual software artifacts~\cite{runeson2007detection,marcus2005recovery,antoniol2002recovering,mccarey2006recommending}.
One common and simple technique is the Vector Space Model~\cite{schutze2008introduction}.
We investigate when text similarity is a suitable technique to retrieve information chunks from build logs.
In the following we will explain the concept of how we apply text similarity to information retrieval from CI build logs, which we refer to as \emph{CTS}\@.

\paragraph{Configuration}
To configure chunk retrieval though text similarity we chose to use the same concept of examples as for PBE\@.
The lines of the output strings of the training examples define our search query.
The algorithm splits the search query into single lines and identifies tokens, in our case words.
Then we build a document-term-frequency matrix over the lines from the search query and prune very often or very rarely appearing words.
Next, the algorithm applies TF-IDF to the matrix, a best practice for natural language queries~\cite{lee1997document}.

\paragraph{Application}
To retrieve the desired information from a build log, we parse the whole text and process it in the same way as the output of the training examples.
The algorithm calculates the cosine similarity~\cite{korenius2007principal} to compare each line of the build log with each line of the search query.
After summing up the similarities of each build log line to all search query lines, we sort the build log lines in decreasing similarity.
The average number of lines in the outputs of the training examples determines how many of the most similar lines are returned as the output of the retrieval run.

\subsection{Keyword Search (KWS)}
\label{sec:expl-skws}
When developers are looking for a specific piece of information within a large amount of unstructured information, a first ad-hoc approach they use is searching for related keywords.
Indeed, this was one of the most common approaches we took when searching for the reason the build failed within a log while creating our \emph{LogChunks} data set.
As this is a technique readily available in many tools developers use to view build logs, we study when such a keyword search is suitable for retrieving information chunks from CI build logs.
In the following we will explain how we use simple keyword search to retrieve information from CI build logs, which we refer to as \emph{KWS}\@.

\paragraph{Configuration}
A set of keywords configures the chunk retrieval with KWS\@.
To better compare KWS with PBE and CTS, we also configure it through examples.
We link each example with keywords, which appear in the targeted chunk or close to it in the input build log.
The configuring keywords for KWS are the ones that appear most often in the keywords of all training examples.

\paragraph{Application}
For a retrieval run, we take a whole build log file as input and search for all exact occurrences of the keywords.
As keywords are often not directly describing the desired information, but rather appear close to the desired information, KWS also retrieves the lines around the found keyword.
The number of surrounding lines retrieved is the average of lines in the output of the training examples.

\subsection{Other Techniques}
\label{sec:expl-rlr}
Literature mentions further build log analysis techniques.
This section describes them with our notion of chunk retrieval techniques.

\paragraph{Log Diff}
Amar et al.\ use a technique based on line diffs and information retrieval to identify relevant lines from a failed build log~\cite{amar2019mining}, as we describe in more detail in Section~\ref{sec:rw-bl-analysis}.
The configuration for the technique is the log from the last successful build and relevant past failures.
This technique retrieves the lines from a build log that are not present in the successful build log and contain terms related to the given past failures.

\paragraph{Random Line Retrieval (RLR)}
In our evaluation, we want to compare against a baseline of randomly extracted lines.
The average number of lines in the outputs of the training examples is the configuration for random line retrieval (RLR).
It retrieves this number of lines randomly sampled from the build log.

\section{Tool Implementation}
For our comparison study we implemented PBE, CTS, KWS and RLR and a unifying interface.
The unified interface is implemented in Ruby and calls the separate technique implementations over the command line.
The implementation of PBE in C\# is based on the Microsoft PROSE library~\cite{prose2019webpage}.
We implemented CTS, KWS and RLR using R and the text2vec library~\cite{text2vec2019webpage}.

\end{document}
