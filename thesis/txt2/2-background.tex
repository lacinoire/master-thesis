\providecommand{\myrootdir}{..}
\documentclass[\myrootdir/main.tex]{subfiles}

\begin{document}

\chapter{Background}

\section{semi-structured data}
characteristics of semi-structured data (structure irregular, implicit, partial, rapidly evolving) and explaining why this applies to buildlogs, extract information to build a structured layer above the unformed data for efficient data access (for developers here! -> impact on our work with buildlogs) \cite{abiteboul1997querying}

maybe: reference \cite{smith1997information} and explain approach of first structuring into sections / "contexts" before extracting acutal information. but only if we actually do that and not just always parse the whole log.

program synthesis from example, neccessary theoretical stuff,
here all the references to prose theoretical papers \cite{le2014flashextract:}, \cite{polozov2015flashmeta:}, \cite{mitchell1982generalization}, 


\end{document}
