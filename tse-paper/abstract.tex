Build logs are a textual by-product that automatic software build processes
create, often as part of their Continuous Integration (CI)
pipeline.
Build logs can serve as a paramount source of
information for developers to understand and debug depdency, compilation, or
test failures.
Manually
extracting the important chunks of information, though, is akin to
finding a needle in a haystack.
Recently, researchers and practitioners have started attempts to partly
automate
this time-consuming activity, with the proposition to not only ease
developers' and researchers task of finding this needle but also to enable a novel class of
on-ward processing applications.
In this paper, we give the first systematic overview of the emerging field of build log analysis.
In a large-scale literature mapping study, we first survey and categorize the existing methods to
extract information from build logs.
In the second part of the paper, we then develop prototypical implementations of the three most promising
techniques, namely program synthesis
by example, textual similarity, and keyword search.
Finally, we evaluate the techniques in an empirical study on
the manually
labeled \emph{LogChunks} data set, which comprises 797 build logs 
from a wide range of build environments.
Our findings show that none of the three techniques in general outperforms
the others, but rather, that each technique has its respective strengths
and weaknesses.
We discuss under which circumstances each technique performs best
and provide a recommendation scheme for when developers or researchers might 
use which technique.
